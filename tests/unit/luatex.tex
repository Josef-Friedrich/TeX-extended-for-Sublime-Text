%!TEX program = lualatex

\directlua{
  a = "aeiou" -- a string
  b = 13 -- a number
  c = function() -- a function
    print ("\n\n\tAin't it grand")
  end

  function printit(tata)  -- print their types.
    for key, value in ipairs(tata) do print(key, type(value)) end
  end
  printit(d)
}


\latelua{
local function Ordered()
  -- nextkey and firstkey are used as markers; nextkey[firstkey] is
  -- the first key in the table, and nextkey[nextkey] is the last key.
  -- nextkey[nextkey[nextkey]] should always be nil.

  nextkey[nextkey] = firstkey

  local function onext(self, key)
    while key ~= nil do
      key = nextkey[key]
      local val = self[key]
      if val ~= nil then return key, val end
    end
  end

  -- To save on tables, we use firstkey for the (customised)
  -- metatable; this line is just for documentation
  local selfmeta = firstkey

  -- record the nextkey table, for routines lacking the closure
  selfmeta.__nextkey = nextkey

  -- setting a new key (might) require adding the key to the chain
  function selfmeta:__newindex(key, val)
    rawset(self, key, val)
    if nextkey[key] == nil then -- adding a new key
      nextkey[nextkey[nextkey]] = key
      nextkey[nextkey] = key
    end
  end

  -- if you don't have the __pairs patch, use this:
  -- local _p = pairs; function pairs(t, ...)
  --    return (getmetatable(t).__pairs or _p)(t, ...) end
  function selfmeta:__pairs() return onext, self, firstkey end

  return setmetatable(key2val, selfmeta)
end
}

\luadirect{
function show(t)
  print (t.__name .. ":")
  for k, v in ordered(t) do print(k, v) end
end

tab.firstName = "Rici"
tab.lastName = "Lake"
tab.maternalLastName = "Papert"
show(tab)
}

\luaexec{
local function node(name, left, right)

end

local tree = node(
  'cow',
  node('rat', 1, 2),
  node('pig', node('dog', 3, 4), 5)
)
}

\begin{luacode}
local Vector = {}
Vector.__index = Vector

function Vector:new(x, y, z)    -- The constructor
  return setmetatable({x = x, y = y, z = z}, Vector)
end

function Vector:magnitude()     -- Another method
  -- Reference the implicit object using self
  return math.sqrt(self.x^2 + self.y^2 + self.z^2)
end

local vec = Vector:new(0, 1, 0) -- Create a vector
print(vec:magnitude())          -- Call a method (output: 1)
print(vec.x)                    -- Access a member variable (output: 0)
\end{luacode}

\begin{luacode*}
Point = {}

Point.new = function(x, y)
  return {x = x, y = y}  --  return {["x"] = x, ["y"] = y}
end

Point.set_x = function(point, x)
  point.x = x  --  point["x"] = x;
end
\end{luacode*}



